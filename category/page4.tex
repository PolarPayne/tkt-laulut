\headerfooteroff{}
\pagecolor{black!5}\afterpage{\nopagecolor}
\lstset{ %
  basicstyle=\footnotesize\tt,        % the size of the fonts that are used for the code
  %breakatwhitespace=false,         % sets if automatic breaks should only happen at whitespace
  breaklines=true,                 % sets automatic line breaking
  captionpos=b,                    % sets the caption-position to bottom
  commentstyle=\color{black!30},    % comment style
  escapeinside={\%*}{*)},          % if you want to add LaTeX within your code
  extendedchars=true,              % lets you use non-ASCII characters; for 8-bits encodings only
  %frame=single,	                   % adds a frame around the code
  keepspaces=true,                 % keeps spaces in text, useful for keeping indentation of code
  keywordstyle=\color{black!60},       % keyword style
  %language=Octave,                 % the language of the code
  morekeywords={*,...},           % if you want to add more keywords to the set
  %numbers=left,                    % where to put the line-numbers; possible values are (none, left, right)
  numbers=left,
  xleftmargin=1.5em,
  framexleftmargin=1em,
  numbersep=5pt,                   % how far the line-numbers are from the code
  numberstyle=\footnotesize\color{black!60}, % the style that is used for the line-numbers
  rulecolor=\color{black},         % if not set, the frame-color may be changed on line-breaks within not-black text
  showspaces=false,                % show spaces everywhere adding particular underscores; it overrides 'showstringspaces'
  showstringspaces=false,          % underline spaces within strings only
  showtabs=false,                  % show tabs within strings adding particular underscores
  stepnumber=1,                    % the step between two line-numbers. If it's 1, each line will be numbered
  stringstyle=\color{yellow},     % string literal style
  tabsize=2	                   % sets default tabsize to 2 spaces
  %title=\lstname                   % show the filename of files included with \lstinputlisting; also try caption instead of title
}
\begin{lstlisting}[language=C]
#include <stdio.h>
#include <vector>
#include <string>

struct DankCharacter
{
	int character;
};

int main()
{
	std::vector<DankCharacter*> charContainer;
	int magicalOffsets[48] = { 1, 17, 9, 17, 5, 32, 19, 2, 1, 1, 13, 32, 19, 21, 9, 17, 32, 12, 2, 8, 32, 8, 3, 32, 1, 17, 9, 17, 5, 32, 19, 2, 1, 1, 13, 32, 24, 17, 7, 32, 12, 2, 8, 32, 16, 2, 10, 1 };
	for (int ch : magicalOffsets)
	{
		auto pCharacter = new DankCharacter;
		ch = ch == 32 ? 0x20 : ((ch + (25 - 13)) % 25) + 0x61;
		__asm__(
			"mov %%ecx, dword ptr[%1]\n"
			"mov dword ptr[%0], %%ecx\n"
			:
		: "r"(pCharacter), "r"(&ch));
		charContainer.push_back(pCharacter);
	}

	for (int i = 0; i < charContainer.size(); ++i)
		printf(i == (charContainer.size() - 1) ? "%c\n" : "%c", (char)charContainer[i]->character);
	for (int i = 0; i < charContainer.size(); i++) delete charContainer[i];
	return 0;
}
\end{lstlisting}
\clearpage
\headerfooteron{}
\fancyhead[CE]{C++}
