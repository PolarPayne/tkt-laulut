\headerfooteroff{}
\pagecolor{black!5}\afterpage{\nopagecolor}
\lstset{ %
  basicstyle=\footnotesize\tt,        % the size of the fonts that are used for the code
  %breakatwhitespace=false,         % sets if automatic breaks should only happen at whitespace
  breaklines=true,                 % sets automatic line breaking
  captionpos=b,                    % sets the caption-position to bottom
  commentstyle=\color{black!30},    % comment style
  escapeinside={\%*}{*)},          % if you want to add LaTeX within your code
  extendedchars=true,              % lets you use non-ASCII characters; for 8-bits encodings only
  %frame=single,	                   % adds a frame around the code
  keepspaces=true,                 % keeps spaces in text, useful for keeping indentation of code
  keywordstyle=\color{black!60},       % keyword style
  %language=Octave,                 % the language of the code
  morekeywords={*,...},           % if you want to add more keywords to the set
  numbers=left,                    % where to put the line-numbers; possible values are (none, left, right)
  numbersep=5pt,                   % how far the line-numbers are from the code
  numberstyle=\footnotesize\color{black!60}, % the style that is used for the line-numbers
  rulecolor=\color{black},         % if not set, the frame-color may be changed on line-breaks within not-black text
  showspaces=false,                % show spaces everywhere adding particular underscores; it overrides 'showstringspaces'
  showstringspaces=false,          % underline spaces within strings only
  showtabs=false,                  % show tabs within strings adding particular underscores
  stepnumber=1,                    % the step between two line-numbers. If it's 1, each line will be numbered
  stringstyle=\color{yellow},     % string literal style
  tabsize=2	                   % sets default tabsize to 2 spaces
  %title=\lstname                   % show the filename of files included with \lstinputlisting; also try caption instead of title
}
\begin{lstlisting}[language=C]
dvec3 sample(dvec3& src, dvec3& dir) {
  if (d > MAXDEPTH) return dvec3(0,0,0);
  isect is;
  trace(src, dir, is);
  if (is.dist == 1e10) return is.color;
  // intersection coord
  dvec3 intersect = dvec3(src + dir * is.dist),        
        r         = dvec3(dir + is.ref);
  if (is.mat == DIFFUSE) {
    dvec3 obj_color = is.color;
    dvec3 obj_ref   = is.ref;
    dvec3 c(0,0,0);
    for (sphere light : lights) {
      double lx = light.p.x + SOFT_SHADOWS,
             ly = light.p.y,
             lz = light.p.z + SOFT_SHADOWS;
      // direction to light
      dvec3 lightdir = norm(dvec3(lx, ly, lz));
      dvec3 lc = obj_color;
      // lambertian factor
      double b = dot(lightdir, obj_ref);
      if (b < 0) continue;
      lc *= pow(b, 3);
      // apply light color
      trace(intersect, lightdir, is);
      lc *= is.light;
      c += lc;
    }
    return clamp(c, 0.0, 1.0);
  } else if (is.light != dvec3(0,0,0)) {
    return is.light;
  } else if (is.mat == REFLECTIVE) {
    return clamp(sample(intersect, r, d+1));
  } else { // is.mat == REFRACTIVE
\end{lstlisting}
\clearpage
\headerfooteron{}