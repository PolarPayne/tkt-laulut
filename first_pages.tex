\noindent\textbf{Tämän kirjan omistaa:}
\\
\\
\\
$\cdots\cdots\cdots\cdots\cdots\cdots\cdots\cdots\cdots\cdots$
\\

\newpage

\textbf{Julkaisija:}

Tekis
\\


\textbf{Laulukirjavastaavat:}

Julius Uusinarkaus

Maximillian Remming
\\

\textbf{Taittosofta:}

PS

Max
\\

\textbf{Lauluja keräsi:}

Felix, Julius, Max, jne, jne 
\\

\textbf{Suurkiitokset:}

Sheets, Python, \LaTeX
\\

\textbf{Painos:}

5/2017 - 520 kpl

\newpage

\afterpage{\blankpage}
\begin{figure}[h!]
\centering
\includegraphics[scale=0.4]{graphics/logo.png}
\end{figure}

\newpage

\textbf{Esipuhe}
\\

Matemaatikot latoivat laulukirjan käsin latexilla, perinteitä kunnioittaen.
Valtiotieteilijät toisaalta pitkillä kokouksilla. Itse kunnioitimme
tietojenkäsittelijöiden perinteitä. Laulu\-kirjamme on ladottu excelistä
python-scriptillä \LaTeX{}iin. Sopivalla määrällä mee\-mi\-taikaa. Niin, ja tätä on työstetty 6 vuotta.

Laulukirja on paljon enemmän kuin laulukirja. Kirja on tieten\-kin hyvä työkalu
sitseillä, mutta sen aito tarkoitus on jäädä ra\-por\-tik\-si menneestä. Laulukirjaan
kerääntyy satoja viestejä, rivouksia ja muita tervehdyksiä, nykyisiltä ja
menneiltä ystäviltä. Laulukirja kuluu, mutta paranee iän myötä. Teimme tämän
laulukirjan sel\-lai\-sek\-si, että siitä on sinulle vielä iloa pitkään. Siitä tulee
jokaisen sitsin jälkeen enemmän omannäköisesi. Voit lisätä puuttuvat lempi\-laulusi DLC-sivuille.

We wanted to make the book accessible, and actually useful, even if you don’t
speak finnish. We put together as many “fun to drink together, or while drunk”-songs
in english as we could. We tried to make this the ultimate combination of a
sophisticated traditional song book, and meme magic. I think we did well.
\\
\\
Hyvää vappua!

\vfill
\textit{Julius ja Max}

\newpage
\blankpage
\blankpage

